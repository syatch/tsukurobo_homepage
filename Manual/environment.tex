\section{環境構築}
Webページの生成にHUGOという静的サイトジェネレータを使用する.
また,デザインにTellaというHUGOのテーマを使用する.
この章ではこれらの導入手順を示す.

\subsection{Hugoのインストール}
以下のページに従い,HUGOをインストールする.

Install HUGO : \url{https://gohugo.io/getting-started/installing/}

Ubuntuでのインストール手順は次項に記載.

\subsubsection{UbuntuへのHUGOのインストール}
以下Ubuntu20.04で操作を確認(2022/06)

\paragraph{brewをインストール} \leavevmode
参考 Homebrew : \url{https://brew.sh/}

brewがインストールされていない場合,以下のコマンドを実行する.
\begin{lstlisting}[]
  $/bin/bash -c "$(curl -fsSL https://raw.githubusercontent.com/Homebrew/install/HEAD/install.sh)"
\end{lstlisting}

インストール出来たら,パスを追加する.
\begin{lstlisting}[]
  $echo 'eval "$(/home/linuxbrew/.linuxbrew/bin/brew shellenv)"' >> /home/<user name>/.profile
  $eval "$(/home/linuxbrew/.linuxbrew/bin/brew shellenv)"
\end{lstlisting}

\paragraph{HugoをUbuntuにインストール} \leavevmode
brewでHUGOをインストールする.
\begin{lstlisting}[]
  $brew install hugo
\end{lstlisting}

hugoがインストールされたことを確認.
\begin{lstlisting}[]
  $hugo version
\end{lstlisting}

\subsection{HUGO Themes : Tellaの導入}
すでに済んでいる.備忘のため,以下にURLを記載する.

HUGO Themes:Tella : \url{https://themes.gohugo.io/themes/tella/}

\subsubsection{npmのインストール}

tsukurobo/で以下のコマンドを実行.
\begin{lstlisting}[]
  $npm install
\end{lstlisting}

\subsubsection{ローカルで作成したサイトを確認}

\begin{lstlisting}[]
  $npm run start
\end{lstlisting}

ここで,以下のようなエラーが出る場合がある.

\begin{lstlisting}[]
  $Error: yargs parser supports a minimum Node.js version of 12. Read our(以下略
\end{lstlisting}

node.jsのバージョンが古いのが原因.
以下のコマンドでnode.jsのバージョンを確認.

\begin{lstlisting}[]
  $node --version
\end{lstlisting}

nodejs : \url{https://nodejs.org/ja/}
で最新版を確認.

\begin{lstlisting}[]
  $curl -fsSL https://deb.nodesource.com/setup_16.x | sudo -E bash -
  $sudo apt-get install -y nodejs
\end{lstlisting}

再度実行し動作を確認.

\begin{lstlisting}[]
  $npm run start
\end{lstlisting}

ブラウザで\url{http://localhost:1313/}を開く.

\subsubsection{サイトのビルド}

\begin{lstlisting}[]
  $npm run build
\end{lstlisting}

tsukurobo/public/にビルドされる.